\documentclass[12pt]{article}
\begin{document}
\title{Data Vis Assignment 5.1}
\author{Jack Cassidy \\ Student Number 1432 0816}
\date{\today}
\maketitle

\section{Node-Link Matrix-Graph Comparison}
\subsection{Node-Link}
Nodes are usually represented as point marks, with links encoded as connection marks between nodes to represent relationships. This approach sacrifices space for relational clarity. \paragraph{}

Node-Link diagrams are suited to tasks where the physical structure of the network is particularly relevant. It might represent roads or a computer network, and the links between the nodes can be used to calculate paths, query for adjacent nodes, calculate weighted distances etc.

\section{Matrix-Graph}
A matrix graph encodes nodes and links into a tabular structure. The rows and columns represent nodes and cells in the table represent a link between nodes. This approach weights spacial efficiency over clarity in cases where the graph may be particularly dense. \paragraph{}

Matrix-Graph representations are more suited to tasks that require scaling, as there is less visual clutter taken up by connection marks between nodes as in the Node-Link view.
\end{document}